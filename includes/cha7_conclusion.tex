\chapter{Conclusion}
\label{cha7}

% What did I present
Throughout this dissertation, we explore the use of modular approaches in three subareas in robot grasping and manipulation: grasp planning, manipulation, and reaching motion planning. In our studies, we formulate the grasping and manipulation as learning problems and use modular approaches to tame the difficulty caused by high dimension and non-linearity.
%
%In this dissertation we discuss the application of modular approaches in robot grasping and manipulation. The basic principle is simple: the solution of a complicated task can be modeled better by a combination of a set of sub-solutions, each of which independently in charge of a subtask of the task. The sub-solutions are extracted from demonstrations and encoded with statistical models.
%This modular-learning hybrid approach is particularly useful for robot grasping and manipulation. Tasks involving in this category have frequently changing context and their system dynamics is hard to model analytically.
%In our method, while the imitation learning approach enable us to directly model the solutions without deeply analysing the system dynamics, the modular approach simplify the modeling process by dividing the big solution space to a set of smaller local regions. For frequently changing context, modular approaches have the advantage of being fast adaptive. Further, user-defined tasks is desired for service robots such as domestic robots. Our method provide an user-friendly framework to program robot, even for difficult adaptive control tasks. Different from many other studies on sequential motor primitives, our modular approaches are ``concurrent'': they are combined as a whole to provide solution for the task. Concurrent combination of modules provides more possibilities of solutions and hence is more feasible.


% ------------- Chapters ---------------
In Chapter 1, we give an overview of the applications of modular approaches in AI, control and robotics, and explain the motivation of using modular approaches in grasping and manipulation. We then further discuss the studies in the relative areas in Chapter 2. We particularly look into the state of art of modular approaches in robot grasping and manipulation.

In Chapter 3, we present our work in real time grasp planning. Two scenarios are considered: grasping known objects and grasping novel objects. For the first scenario, we generate training grasps for a given robot hand and a given object, and learn a GMM to encode the stable grasp distribution. After the grasps of the object being ``learnt'', new grasps can be quickly computed using the model. For the second scenario, we adopt a modular approach based on the concept of shape primitives. The novel objects are regarded as a combination of ``learnt'' shape primitives, and its grasp distribution is formed by combining the corresponding primitives' grasp distributions. Grasps for the novel object are then computed from the distribution. We implement this method on two different robot hands and show that the computation time is no more that 20 $msec$. This method enable the robot to react quickly in robot-human interaction and adapt to fast perturbations in a dynamic environment, and is hence suitable for a service robot. We show that modular approach can speed up solving high-dimensional planning problems.

In Chapter 4, we present a multiple model adaptive control strategy for a manipulation task and use it in a opening bottle task. After recording a human demonstrating the task in different contexts, we perform modular decomposition of the control strategy, using phases of the recorded actions to guide segmentation. Each module represents a part of the strategy, encoded as a pair of forward and inverse models. All modules contribute to the final control policy; their recommendations are integrated via a system of weighting based on their own estimated error in the current task context. We show that our approach can
modularize an adaptive control strategy to generate appropriate motor commands for the robot to accomplish the opening bottle cap task.

In Chapter 5, we present a method to encode motion primitives for reaching motion using mimesis model. The mimesis model enables a robot to recognize and generate motion primitives, as well as symbolize and store them. This method will allow the robot to understand human commands and modulate their behaviours according to the commands. In the experiments of bimanual lifting boxes task, we show that new motion primitives can be generated by combining existing motion primitives with appropriate weighting to successfully lift boxes with different sizes and positions. This method simplifies the modeling and modulation of motion primitives.

In Chapter 6, we discuss the advantages, as well as the limitation of modular approaches. We also suggest a few possible future works following the studies presented in this dissertation.

This dissertation contribute to grasping and manipulation by proposing a few modular approaches to deal with the high dimension and nonlinear problems. The generality of the modular approaches is shown by their use in grasp planning, object manipulation and reaching motion planning. It also contributes to human robot interaction by combining the modularity of motion primitives and of human language to form an understanding base between human and robot. With a library of symbolized motion primitives, it will be able for the robot to recognize human motion, understand and follow human commands. Therefore we conclude that modular approaches are an effective methodology in building intelligence for service robots.
%In this chapter, we show that with the mimesis model works effectively in generating motions various in one dimension (object size, object height). In the future work, we will further study the control in multiple dimensions, using interpolation between multi-proto-symbols.





% What did I contribute
