\section{Discussion}
\label{cha5:sec4}

The system we present in this chapter uses the mimesis model to learn motion primitives for object manipulation. It provides an easy to use interface for robot motion generation. Motion primitives are the elementary motions that accomplish basic functions. Recent brain research~\cite{bizzi2008combining} provides more evidence to support the hypothesis that, in order to reduce the degree of freedom, the vertebrate motor system generates motions by combining a small number of motor primitives. Our experiment shows that by combining two motion primitives (8 d.o.f) we can indeed generate many different motion patterns. This framework largely simplifies the modelling of motion primitives.

From the functional point of view, the motion primitives form the vocabulary of motions. This naturally allows us to label motion primitives using words. In our system, each motion primitive is symbolized in the proto-symbol space and labelled by its effects, i.e. ``grasp high box", ``grasp low box" etc. This provides a linguistic interface for the human to instruct robot motion, by giving language instructions like ``go lower to grasp the box". By matching the words in the human instruction and the labels of the motion primitives, the robot will be able to adjust the mixing coefficient and generate new motions to execute the command.

We implemented the system in the Webots simulator with the iCub robot. The interpolation of the proto-symbols produces new motion primitives. The correlation between the new motions and their effects are learnt using first order linear regression. In our future work of learning more complex motion primitives, higher order or non-linear regression may need to be employed.

The experiment presented here provides a good starting point for our future study in learning more motion primitives for object manipulation. It shows that it is possible to directly control the outcome of the motion pattern, without fine tuning different variables in the model. This has the advantage, from the users point of view, that planning the motion primitives can be achieved more intuitively. In this chapter, we show that with the mimesis model control in one dimension (object size, object height) works effectively \textcolor{red}{what works effectively, the system?}. In the future work, we will further study the control in multiple dimensions, using interpolation between multi-proto-symbols.
