\chapter{Introduction}
\label{cha1}

\setcounter{page}{1}
\pagenumbering{arabic}
%%% Move 1: Establishing a research territory (why interesting)
%%%- by showing that the general research area is important, central, interesting, problematic, etc. (optional)
Grasping and manipulation are essential skills for service robots. Equipped with these skills, robots would be able to provide great assistance to humans in many aspects of daily life from hospital to household environments. Grasping and manipulation has been extensively studied for more than two decades. In industry, robot grippers have been widely used for fast and accurate operations. Outside industry, however, there is still no universal robust solution for grasping or manipulation in a human dominated environment.
%%% challenges leads to PbD: variety of tasks
%\paragraph{Problem statement} ~\\
% TODO: replicate in part the manipulating ability of the human hand.

The main challenge of robot grasping and manipulation comes from the large variety of tasks and the complicated dynamics of the robot-environment interaction. A versatile service robot is expected to be able to handle many tasks in human daily life, from simple pick-and-place tasks to multifinger dexterous manipulation tasks like writing and using tools. Different tasks have different instructions and constraints. Programming each of them by hand coding is painstaking. Further, grasping and manipulation are contact tasks, for which handling contacts between the robot end-effector and the environment is essential. The dynamics of the contacts are usually complicated and involve the study of friction and materials. An analysis of the dynamics of contact tasks requires both a deep understanding of the task, the mechanics of the robot and control theory. It is infeasible for the end user to program such tasks.

%%% introducing and reviewing items of previous research in the area (obligatory)
To tackle this problem, robot learning has been proposed as an alternative to an analytical solution. Learning by demonstration (also called imitation learning and programming by demonstration) has been extensively studied as a promising and user-friendly approach to build robot intelligence~\citep{schaal2003computational,dillmann2004teaching,billard2006discriminative,calinon2007incremental}. This approach is data-driven. It aims to program the robot to extract the success pattern of a particular task from the demonstration data (either from teaching or self-exploration) and to encode this pattern. This approach allows us to model strategies for tasks without deriving the complex dynamics of the environment. The strategies are usually encoded by statistical models allowing certain level of noise. It is particularly useful for tasks where analytical expression of the system is hard to derive, such as contact tasks.


%%% Move 2: Establishing a niche (what is missing in current research)
%%% - by indicating a gap in the previous research or by extending previous knowledge in some way (obligatory)
Although the learning by demonstration approach provides a user-friendly method for the end-user to program robot, learning grasping and manipulation tasks is still challenging. Even for the same task, the planning or control strategy can be different according to the task context. A single model is not adequate for these tasks.

%%% Move 3: Occupying the niche (your work)
%%% - by outlining purposes or stating the nature of the present research (obligatory)
%%% - by listing research questions of hypotheses
%%% - by a statement of your objectives
%%% - by announcing principal findings
%%% - by stating the value of the previous research

%%% What do we do? - modular approach
In this thesis, we propose an approach to further reduce the task complexity: the modular approach.
This approach focuses on the problem of decomposing a complex task into small subsections and developing solutions for each subsection separately. These solutions are then recombined to provide an integrated solution of the task. The benefit of this approach is that it translates a complex problem into many smaller problems, the solutions of which are easier to find.

%%% Benefit of modular approach
%TODO: very few real physical systems are really linear
The modular approach is particularly suitable for tasks involving different contexts or requiring multiple strategies. While switching between multiple modules allows the robot to quickly adapt to a changing environment, combining the modules allows the robot to generate new skills to adapt to new contexts. We apply this approach to the problem of grasping and manipulation tasks, to simplify the learning problem of contact tasks and to build an easy-to-use interface for teaching a robot. This dissertation introduces different ways to modularize tasks and then to combine the modules to accomplish the tasks. It provides a framework to model the modules as statistical models via a learning approach. The work shows that the modular approach in robot grasping and manipulation is not only theoretically attractive but also a practical method.

%%% Introduction structure
In the next section, we provide a brief overview in Section~\ref{cha1:modular} of the use of the modular approach in robotics. We first show the study of modularity in artificial intelligence (AI) and control theory and then show the application of modularity in robotics as the intersection of those two realms. In Section~\ref{cha1:contribution} and~\ref{cha1:organization}, we outline the contributions of this dissertation and present its organization.


%Program by demonstration (also named as learn by demonstration and imitation learning) has been extensively studied~\citep{calinon2007learning,dillmann2004teaching,kulic2012incremental} as a promising approach to build robot intelligence. %It is essential for the tasks that analytical expression of the system is hard to derive.
%Learning manipulation tasks is one of the main application of this approach. The physical properties of a manipulation task is hard to express analytically, and as a result the control strategy is hard to derive. Modeling expert's demonstration of strategies has been used as an alternative to the analytical solution.


%The current existing approaches of learning are task-dependent. Grasping and manipulation tasks, however, are still remained as a challenge problem to learn, due to the complex interaction between the robot and the environment.
% Challenge: dynamics: fast; uncertainty: adapt, human instruction;
%Human dominated environment is clutter and dynamic. This unavoidably produces a large amount of uncertainty and hence make the planning very difficult. Robots working in this environment have to be able to quickly make action plan and adapt to environment changes. At the other hand, because of the large variety of the daily life objects and tasks, the search space is huge. It this impossible to list all the possible states of the environment, e.g. all possible objects for grasping and manipulation, and develop an deterministic strategy by conventional planning and control approaches.

\section{Existing modular approaches}
\label{cha1:modular}
Robotics is an interdisciplinary area. It is an intersection of many fields in engineering and cognitive science. Two of the most important fields in robotics are AI and control theory. While AI concentrates on the high level perception and action planning, control theory focus on robustly and stably delivering the robot to the desired state. Modular approaches have been independently studied in these two areas and shown to be effective for developing autonomous and intelligent systems.

\paragraph{Modularity in AI}
%TODO: Hypothesis of modularity
AI is a field of studying how to enable machines to have animal level intelligence~\citep{brooks1991intelligence}. Modular approaches in AI are inspired by two factors: the evidence of modularity in cognitive science and the efficiency of the modular approach in software engineering.
As a research area that aims to produce animal level intelligence in machines, a branch of AI studies the source of the intelligence, e.g. neuroscience and psychology, and tries to mimic the mechanisms. In both neuroscience and psychology, evidence shows that brain and mind have some modularized structures~\citep{fodor1983modularity,peretz2003modularity,barrett2006modularity,sztarker2011brain}. It is suggested that the modularity in brain and mind helps animals to organize the functionalities and handle complex situations. This evidence motivates researchers in AI to develop modular architectures for machine intelligence. On the other hand, from the software engineering point of view, a modular approach is an effective way of building large complex systems. It is widely used for separating the functionality of a program into independent modules, such that each contains everything necessary to execute only one aspect of the desired functionality. Therefore building a complicated intelligence system inevitably prefers a modular approach. Many forms of modularity have been proposed to study different aspects of AI~\citep{bryson2004modular}.

%Modular approach is widely used in software engineering for separating the functionality of a program into independent modules, such that each contains everything necessary to execute only one aspect of the desired functionality. This is proven to be an effective method to develop large and complex system.......
%Modular approach in AI is inspired from neuroscience, psychology and software~\citep{bryson2004modular,BrysonMcG12}.

\paragraph{Modularity in control}
%%% purpose of modularity in control: adaptive
Modular approaches are used in adaptive control and their benefit has been long discussed~\citep{jacobs1991adaptive,narendra1997adaptive}. They are used to solve the control problem in a dynamic environment, where changes can happen rapidly or discontinuously.
Classic adaptive control approaches such as model identification~\citep{khalil2004modeling} are inadequate for these environments, as instability or error may occur during the optimization of the model variables. \textcolor{red}{To quickly adapt, the multiple model approach referred as modular approach here has been proposed by ~\citet{narendra1995adaptation}.}
In this approach multiple controllers are designed, each of which is in charge of a certain task context. During control, the task context is estimated online and the corresponding controllers are activated. When the task context changes, the system automatically switches to another strategy that is suitable for handling the current context. This ensures that the system reacts quickly enough to adapt to the environment.
%In tasks, context changing is a common phenomenon due to object interactions. These changes are often rapid or discontinuous.
%Some recent work~\citep{fekri2007robust,kuipers2010multiple} presents promising modular based approaches.
%These  especially for those working in dynamic environment where changes and adaption are essential.

\paragraph{Application of modular approaches in robotics}
Briefly speaking, modular approaches in AI mainly target decomposing tasks to simplify the design of agents, while control theory mainly aims to build a fast adaptive control policy. In robotics, modular approaches are used for both of these two purposes. Roboticists usually focus on more specific tasks, such as grasping and walking, and try to develop robust and stable plans to accomplish those tasks. \textcolor{red}{In fact, the divergence of the research interests, e.g. grasping and walking, is itself a modular approach: the high level modularity divides the research community into different interest groups that each try to provide a generic solution for a specific task.}

Further, even for the same research interest group, modular approaches are used to reduce the complexity of design and increase the flexibility of the planning. Some of the most well known modular approaches in robotics use motion primitives for motion planning~\citep{ijspeert2002movement,inamura2004embodied,kulic2008incremental,peters2008reinforcement}, hand synergies~\citep{santello2000force,gabiccini2011role,gioioso2013mapping}, eigen-grasp~\citep{Ciocarlie2009} and grasp by shape primitives~\citep{miller2003automatic,huebner2008minimum} for grasp planning and etc.



~\\
In conclusion, modular approaches are widely used in robotics. They are mainly used to tame the complexity of high level task planning and low level strategy selection. However, how to modularize a task in order to facilitate robot learning is rarely discussed in literature and remains an open problem.






\section{Our modular approach in robot grasping and manipulation}
\label{cha1:contribution}
%outline of our approach
%what is module
The definition of a module varies by discipline. Here we define a module as a functional unit that takes certain inputs and provides certain outputs. The computation from the inputs to the outputs is independent to other units. Although the concept of modularity in cognitive science is still in debate, its efficiency in software design is well recognized. In this thesis, we do not try to argue the role of modularity in animal brains. We simply take the concept and exploit its effectiveness in programming robots to carry out tasks. The tasks we discuss here are primitive tasks that can be described by a simple language such as ``grasp'' and ``open'' and no further subtask needs to be decomposed. Therefore the modularity we study is task-specific: multiple modules serve one task and each module serves one task context. We hence call our modularity ``task level modularity''. Not all primitive tasks are in need of a modular approach. Some simple tasks such as ``close your eyes'' have a generic solution. However, in grasping and manipulation, this is usually not the case. As discussed before, the contacts between the robot end-effector and the environment makes the system hard to analyze and the large variety of tasks makes it hard to find a universal solution. In our studies, we explore a few possible ways to use a modular approach to tame these problems.

%Different from the high level modularity, we focus on the low level modularity: deep modularity. While high level modularity is about modules specializing in different functions such as vision, tactile and speech, our study in deep modularity concentrates on modules working for the same function but working in different task contexts: it is about modularity in modularity. We study how to design these most fundamental modules and combining them to achieve higher level task.

We apply the modular approach in the three main domains of grasping and manipulation: grasp planning, manipulation force control and reaching. These three tasks have different challenges and require different modularization methods. For grasp planning, we modularize the strategy by the object shape and propose a method to quickly plan grasps for novel objects. For manipulation, we modularize the control policy by task context and equip the robot with human level adaptive skills. For reaching, we modularize the movement by human command, which builds an understanding base between robots and humans by language and allows the human user to easily teach robot new motion primitives.

%Each of them use different approach to modularize the task: by the target object shape, by the task context and by the human command. Them each focus on different aspects of modular approach: how to combine different modules, how to decompose the task and how to modulate the modules.
These three approaches enable the robots to accomplish tasks that are complex but can be pre-planned, need to adapt in real time, or need to follow human instructions.


%TODO: 3 works. connection.

%Grasp planning: what need modular; benefit of
\paragraph{Grasp planning: modularize by perception (Chapter~\ref{cha3})}
The first contribution is modularity in multifinger grasp planning. Previous research in robot grasping focus on synthesizing grasps analytically, using precise and accurate models for the objects~\citep{sahbani2011overview}. Those approaches are usually computationally expensive for the high degree of freedom of the multifinger robot hand and the universal representation of the object, which usually have many variables. To tackle this problem, we modularize grasping by the shape of the objects. \textcolor{red}{In our work, we first focus on fast generation of grasps for familiar objects and then extend the approach to generate grasps for novel objects.}
Initially, we learn the statistical model for the feasible grasps of a familiar object. This distribution is then used to quickly generate grasps. A novel object can then be represented as a compound of shape primitives, e.g. sphere, cylinder and box. The grasp distribution of these shape primitives are pre-trained and each acts as a module. We combine the grasp distributions of the shape primitives to form a new grasp distribution for the novel objects. \textcolor{red}{When combining, the overlapping and conflicting regions between shape primitives are excluded.}
This approach does not require a general and accurate representation of the object. As grasps can be planned quickly, fast correction can be done for small modelling error. The first part of the work is published in ICRA 2013~\citep{bidan2013grasp}.

%In this work we take the approach of grasp by shape primitive~\ref{miller2003automatic} and focus on how to fast generate grasps for a novel object from its shape primitive components. As mentioned above, robot grasp planning is a difficult problem. Studies show that human tame this problem by choosing preshape postures of the hand according to the object shape primitive components, i.e. sphere, cylinder and box. Inspired by this, we build models for hand postures specialized for grasping the shape primitives so as that grasps can be quickly generated from the models. We further develop a method to combine these models to form a new model for unseen objects. The biggest benefit is to plan grasp in real time. Real time planning strategy is crucial for robots working in dynamic environments. In particular, robot grasping tasks require quick reactions in many applications such as human-robot interaction. This approach enables robots to plan new grasps rapidly according to the object position and orientation. This is achieved by taking a three-step approach. In the first step, we compute a variety of stable grasps for a given object. In the second step, we propose a strategy that learns a probability distribution of grasps based on the computed grasps. In the third step, we use the model to quickly generate grasps.

%Manipulation
\paragraph{Dexterous manipulation: modularize by action (Chapter~\ref{cha4})}
The second contribution concerns manipulation. Object manipulation is a challenging task for a robot as the complicated physics involved in object interaction is hard to express analytically. In this work we introduce a modular approach to learn the human manipulation strategy. After a human demonstrates a task in different contexts, we modularize the control strategies according to the contexts. Strategy in each module is encoded by a pair of forward and inverse models. All modules contribute to the final control policy, according to their estimation errors of the current task context. We validate our approach on a robot platform with a task to open a bottle cap. We show that our approach can modularize the adaptive control strategy to generate appropriate motor commands for the robot to accomplish the task. Fast estimation of the current task context and choice of the correct module enables the robot to react to changes of environment. This work is submitted to the journal Autonomous Robots.


%Motion primitive
\paragraph{Motion primitive: modularize by language (Chapter~\ref{cha5})}
The third contribution concerns learning reaching motion primitives for manipulation tasks. In this work, we develop an easy-to-use human interface for teaching and commanding a robot to carry out manipulation tasks. The human-demonstrated manipulation motion primitives are initially encoded by statistical models. The models are then projected to a topological space where they are labeled by a language description of their properties. We explore the unknown area in this space by interpolation between the models. New motion primitives are thus generated from the unknown area to meet new manipulation scenarios.
Human commands are understood by matching with the labels of the motion primitives. Humans can give new commands during execution to correct improper robot behaviour. Here we make use of the modular nature of human language to modularise robot motion. This work is published in ROBIO 2013~\citep{bidan2013robio}.


%We develop a system to learn manipulation motion primitives from human demonstration. This system, based on the statistical model Mimesis Model provides an easy-to-use human-interface for learning manipulation motion primitives, as well as a natural language interface allowing human to modify and instruct robot motions. The human-demonstrated manipulation motion primitives are initially encoded by Hidden Markov Models (HMM). The models are then projected to a topological space where they are labeled, and their similarities are represented as their distances in the space. We then explore the unknown area in this space by interpolation between known models. New motion primitives are thus generated from the unknown area to meet the new manipulation scenarios.
%We demonstrate this system by learning bimanual grasping strategies. The implemented system successfully reproduces and generalizes the motion primitives in different grasping scenarios.


%TODO: thesis contribution
%This work contributes a framework to modularize human ...


\section{Organization of the thesis}
\label{cha1:organization}
This dissertation has 6 chapters. Chapter~\ref{cha2} gives an overview of existing modular approaches in robotics, discusses its benefits and challenges and describes the framework of our approach. Chapter \ref{cha3} to \ref{cha5} details our work in learning grasp planning, manipulation and reaching motions. We discuss the advantages of our modular approach in grasping and manipulation tasks and the potential to extend it to other areas. Chapter six discusses the achievement of our work and summarizes the contribution.

