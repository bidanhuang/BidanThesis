\section{A review of robot grasping and manipulation}
\label{cha2:sec1}
%TODO: introduction  of grasping and manipulation. What is grasping? What is manipulation?
As discussed in the first chapter, grasping and manipulation problems are important but difficult to solve.
Robot grasping and manipulation research aims to enable robots with a human level ability of handling objects. Grasping and manipulation are usually included in the same research category and are studied by the same robotics community, as they both try to tackle the ``contact tasks'', which  use robot hands (end-effectors) to get physical contacts and interact with target objects.
Robot grasping focuses on how to stabilize the target objects with the support from the robot hand. This involves the problem of where and how to place the hand and fingers to contact the targeted objects. Robot manipulation focuses on delivering the targeted objects from the current state to a desired state, which involves the problem of how to apply forces and torques on the object to achieve the desired state. Besides grasp planning and manipulation, the reaching problem is also frequently discussed in the community.
The problem studied in reaching is how to move the robot hand to reach the object so that the planned grasps or manipulation strategy can be achieved, for example making contacts in the right places to pick up a box. In the later three sections, we will present an overview of these three topics.
%TODO: divide to three sections.
%Reaching: forward and inverse kinematics
%Grasping: analysis and synthesis of closure grasps
%Manipulation: prehensile and nonprehensile

\subsection{Robot grasp planning}
\label{cha2:sec1:planning}
%\label{cha2:sec1:grasping:planing}

% ----- What is grasp planning. Force/form closure -----
The studies of robot grasping have two main categories: geometric based planning and control based execution. The first category studies how to pose the hand and fingers form a stable grasp and the latter studies how to execute a grasp plan and how to make local adjustment to correct a unstable grasp. Early studies of grasping mainly focus on the first category and the second category raise increasing interests in recent years. In this review, we will first look into the planning problem and then move to the execution problem.

Given a robot hand and an object, there are an infinite number of ways to grasp the object. These grasps have different performances and functionalities. Grasp planning is usually formulated as an optimization problem of grasp performance, by finding the contact point locations or robot hand configuration. This technique is call optimal grasp synthesis. The most important criteria in the optimization is the stability of the grasp. In the robot grasping literature, two kinds of ``closure'' are the most extensively used mechanisms for guaranteeing stability: the force-closure and form-closure~\citep{Nguyen87}. A grasp is said to achieve force-closure when the fingers can apply appropriate forces on an object to produce wrench (force and torque) in any direction~\citep{SalisburyJr1985}. Form-closure is a stronger condition than force closure, which can only be achieved if a grasp is force closure with frictionless contact points~\citep{diziouglu1984mechanics}.

% ---- Grasp quality ----
To measure grasp stability qualitatively, the concept of grasp quality is introduced. Various grasp quality metrics are proposed. One important concept involves is the ``grasp wrench space'', i.e. the space of the possible force and torque to be applied by the fingers.
%Starting from the idea of minimizing the sum of the contact forces, \citet{li1988task}, \citet{kirkpatrick1992quantitative} and \citet{ferrari1992planning} propose different measurements of the grasp quality based on the hand wrench space.
The concept of ``task ellipsoid'', a geometric representation of the wrench required in a task, is proposed by \citet{li1988task} to measure how suitable a grasp is for the task: the more overlaps between the task ellipsoid and the grasp wrench space, the more suitable this grasp is.
\citet{kirkpatrick1992quantitative} refer to the grasp quality as the ``efficiency'' of a grasp and define it as the ratio of the largest external wrench that can be balanced by at most one unit force at each contact point. Based on the same principle, \citet{ferrari1992planning} define the quality of a grasp to be the minimum distance from the origin of the wrench space to the edge of the grasp wrench space.
\citet{trinkle1992stability} proposes a test to measure how far is a grasp away from the form closure.
These metrics are ``object-centric'', i.e. they only consider the contact point locations and the object geometry, while the robot hand configuration is not taken into account. \citet{miller1999examples} take one step further: they use a simulation method to compute the grasp quality of a given object and robot hand configuration. They later develop the physical simulator GraspIt! for grasp quality analysis~\citep{miller2004graspit}. Our work in grasp planing described in Chapter~\ref{cha3} is based on this simulator.

% TODO: Some grasp synthesis methods. Add more.
%Optimal force-closure grasp synthesis is a technique for optimizing the grasping performance by finding the contact point locations.
Optimal force-closure grasp synthesis concerns determining the contact point locations so that the grasp achieves the most desirable performance in resisting external wrench loads.
Based on the grasp quality concept, some approaches optimize an objective function according to a pre-defined quality criterion~\citep{Zhu2003,Zhu04} in the grasp configuration space.
%Some techniques compute optimal force-closure grasps by
%optimizing an objective function according to a pre-defined
%quality criterion~\citep{Zhu04, Zhu03} in the grasp configuration space.
These approaches do not take into account the kinematics of the hand. To bridge this gap, \citet{S.ElKhoury2012} propose a one shot grasp synthesis approach that formulates and solves the problem as a constraint-based optimization.

% ----- Learning and modular methods to tame the dimension -----
Multi-finger grasps usually involve a large number of degrees of freedom.
Searching the grasp space for an optimal grasp requires massive computing time considering the huge number of possible hand configurations. To solve this problem, imitation learning and modular approaches are introduced to constrain the searching space. The relevant literatures are reviewed in Section~\ref{cha2:sec2:grasping-learning} and Section~\ref{cha2:sec3:robotics}

% ----- grasping in cluttered environment -----
The above methods are for static grasp planning that rely on precise and accurate object models. These methods are well suited to controlled industrial environments, for example picking up aligned boxes from the assembly line. However, they are not very applicable for service robots working in human dominated environments. For this reason, in recent years the research has shifted to tackle the problem of maintaining grasp stability in dynamic and cluttered scenes. These studies include handling uncertainty and noise in perceptual data and handling unseen (novel) objects and unforseen situations. To tackle the former problem, one approach is to take the uncertainty and noise into account in the planning and generate robust grasps~\citep{brost1988automatic,zheng2005coping,hsiao2011bayesian}.
\citet{brook2011collaborative} try to handle the uncertainties in object shape estimation by finding a common grasp of the few most possible object shapes.
Besides synthesis, grasping motion is also studied~\citep{,kehoe2012toward}, where the uncertainty is handled by the compliant finger motions. For grasping novel objects, different general object shape representations are proposed. The most studied representations are 2D or 3D local features such as edge, contour and color~\citep{saxena2008robotic,detry2009learning,kroemer2010grasping}, local principle curvatures \citep{kopicki2014learning}, combination of shape primitives~\citep{miller2003automatic,huebner2008minimum,el2010new} and exclusive mathematical representation of the global object surface geometry and topology~\citep{el2013generation,pokorny2013grasp}. These features are associated to possible grasps. When a particular feature is identified in a novel object, it is used as a key to query suitable grasps for the object. Among these features and representations, local features allow quick computation of grasps on a sub-part of an object, while global representations allow a global search of good grasps with large computation expenses. Planning grasps for novel objects effectively and robustly remains a challenge.



\subsection{Robot manipulation}
\label{cha2:sec1:manipulation}
%\label{cha2:sec1:grasping:manipulation}

% Current state of art in manipulation

%Different from grasping which aim to stabilize a object, manipulation aim to change the object status, usually its position and orientation, from the current one to the desire one.
Manipulation differs from grasping in that it aims to change the object status, usually its position and orientation, from the current one to the desired one, whereas grasping merely aims to stabilize the object.
This means the problem of manipulation is two-fold: controlling the hand movement to control the object movement.
Studies in manipulation can also be split into two topics: manipulation planning and task execution. The former focuses on planning the hand movement, reasoning how to accomplish a complex task by a sequence of motions and behaviours, while the latter focuses on controlling the object movement, answering the question of how to apply force and torque to deliver a target object to the next desired status. The former problem is mostly addressed by learning from humans and extracting motion primitives from human demonstration, which can be used to build complex behavior for accomplishing a task. We will review those works in Section~\ref{cha2:sec2:grasping-learning} that reviews imitation learning and Section~\ref{cha2:sec3:robotics} that reviews modular approaches. In this section we will concentrate on the latter problem of execution.
%Briefly speaking, there are two directions of approaches: the position and force hybrid control method and the impedance control method.
% Impedance: find a good task impedance.
% force/position hybrid: transition

% open challenge -- to verify your contribution
% why need imitation learning? why need modular?

% Difficulty
%The fundamental difficulty of manipulation is that it requires the robot to adapt to the current situation and tackle sudden changes.
%Manipulation is difficult for that it involves the interaction between robot and the environment.

% TODO: hybrid control
Control methods for manipulation can be roughly divided into two groups: hybrid position $\backslash$ force control and impedance control. The hybrid control approaches directly control the force and the position of the robot hand \citep{li1989grasping,yoshikawa1993coordinated}. It specifies which directions to control the force and which directions to control the position, and control both force and position at the same time. On one hand, this direct control of force and position allows a precise control of the hand-environment interaction. On the other hand, it requires a fast reaction to task context changes, e.g. transition between contact and no contact, and a small delay in control may cause large force overshot.


% Impedance control
% TODO: virtual spring
In the contrast, the impedance control method indirectly control the force via defining impedance of the hand~\citep{howard2010transferring,wimbock2012comparison}. Given the desired impedance of a task, we can compute proper motor commands for the robot to accomplish it. Fixed impedance control is limited to simple tasks. In many manipulation tasks such as opening a bottle cap, variable impedance is required: at the beginning we need a large impedance to break the contact between the bottle and the cap, and later we need a small impedance to drive the cap smoothly. For such tasks fixed impedance control will either lead to task failure or cause hardware damage.
However, computing the impedance for a given task involving variable impedance is difficult.
In many cases the impedance is roughly approximated by a linear model, but this is inadequate for non-linear tasks.

Variable impedance can be learnt by humans physically correcting the robot impedance, i.e. wiggling the robot arm, in different stages of the task~\citep{kronander2012online}. For learning manipulation, however, wiggling the robot fingers will interrupt the task and may cause task failure.
%This is feasible for learning robot arm impedance but not for object impedance.
Variable impedance can also be learnt by the Policy Improvement with Path Integrals ($PI^2$) reinforcement learning algorithm, with a task specific cost function~\citep{buchli2011learning}. Designing this cost function requires insight into the task and is usually difficult.

% TODO: rolling and sliding -- difficult
Most of these control methods assume fix point contacts between robot and the environment. In reality, manipulation control always involves rolling and sliding between the contact surfaces. The dynamics of rolling and sliding are analysed in various of studies \citep{howe1988sliding,montana1988kinematics}. These needs to be taken into account in order to rigorously implement the control methods. However, an analytical model of friction that can reliably predict sliding and can result in stable analysis of the system dynamics is no yet available \citep{bicchi2001robotic}. This makes the manipulation process hard to predict and requires the robot to adapt to the current situation and tackle sudden changes in real time. This inspires us to learn how human adapt to changing contexts and accomplish manipulation tasks. We presents our study in this direction in the Chapter~\ref{cha4}.



\subsection{Reaching motion planning}
\label{cha2:sec1:reaching}
%\label{cha2:sec1:grasping:reaching}

Reaching motion is another key component in the robot grasping and manipulation problem. Given a computed stable grasp, the question to answer in this study is how to deliver the robot hand to the desired position and form the desired hand posture. This is not a simple path planning problem for the robot arm, but a high dimensional planning problem taking the multiple finger movement into account. On one hand, most studies try to plan a motion to avoid premature collisions between the hand and the object. To this end, the finger movement and the arm movement always need to couple in order to ensure the fingers clutch at the right moment \citep{Shukla2011CDS}, and curve around the object to form the desired grasps \citep{kroemer2011grasping}. To increase the robustness of a grasp, the uncertainty in perception is also taken into account \citep{stulp2011learning}. On the other hand, however, some researches study how to deliberately produce ``premature'' contact with the object.

\citet{chang2010planning} study the human ``pre-grasp'' movements such as sliding a coin to the table edge in order to pick it up, and rotating the handle of a pan to a proper position to grasp it. These methods largely increase the chance of successfully executing a grasp by changing the object's status.
