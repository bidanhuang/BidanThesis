\chapter{Discussion and future work}
\label{cha6}

%  advantage of "put the user in the loop of robot's learning"
%  advantage of "modular approach"; different way of modularization
%  advantage of "imitation" + "modular" in grasping and manipulation
%  discuss the use of HMM and GMM? other statistical model?
%  limitation and failures of current system
% suggest direction of research for the future work

The modular approach is an effective way to simply the design of complex system. It is used both in natural intelligence and software engineering.

The modularity we study is the ``task level'', which concentrate on the modularity appear in individual tasks. Its main idea is to simply the programming of robots and provide an end-user level interface to teach robot to do tasks. A future work is to study the ``super-task level'' modularity, i.e. to group the tasks can share same modules together and build modules for them. Eventually, a hierarchy structure of modules will be build. Learn a bank of modules.

Offline learning -> online learning. Identify new task and increase number of modules.

modularization can be done by language.

a framework to do all the modularization: extract, model, choose, sequence, spacial combination, mix 